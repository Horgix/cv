\newcommand\Colorhref[3][color1]{\href{#2}{\color{#1}\underline{#3}}}
\documentclass[10pt,a4paper]{moderncv}

\usepackage{moderntimeline}
\usepackage[french]{babel}
\usepackage[utf8x]{inputenc}
\usepackage[T1]{fontenc}

%\renewcommand{\FrenchLabelItem}{\textcolor{blue}{$\circ$}}

\moderncvstyle{classic}
\moderncvcolor{blue}
\tlmaxdates{2010}{2016}
\tlwidth{0.8ex}
\tltext{\tiny}
% adjust the page margins
\usepackage[scale=0.90]{geometry}
\setlength{\hintscolumnwidth}{2.5cm}           % if you want to change the width of the column with the dates

% personal data
\firstname{Alexis}
\familyname{Chotard}
\title{Ingénieur Système \& Infrastructure\newline
DevOps à Smile Open Source Solutions}
\address{1 rue Jean Mermoz}{Appartement 157}{92320 Châtillon}
\mobile{06-08-92-41-96}
% \phone{03-29-32-71-27}
\email{alexis.horgix.chotard@gmail.com}
\homepage{https://github.com/horgix}
\extrainfo{23 ans - Permis B}
\photo[60pt][0.2pt]{photo_cv}
\quote{"People think that computer science is the art of geniuses but the
actual reality is the opposite, just many people doing things that build on
each other, like a wall of mini stones." -- Donald Knuth}

\begin{document}

\makecvtitle

\section{Expérience}

\subsection{Professionnelle}

\tlcventry{2015}{0}{Ingénieur Système et Infrastructure / DevOps}{Smile Open
Source Solutions}{Paris}{}
{
  \begin{itemize}
    \item Industrialisation : \textbf{Ansible}, \textbf{SaltStack}
    \item Conception d'architectures virtualisées : \textbf{Docker}, LXC,
      Proxmox, OpenVZ
    \item Mise en place d'architectures web hautement disponibles
    \item Génération de plateformes de démonstration via Ansible, Tower et AWS
    \item Tests de montée en charge via JMeter
    \item Formations : scripting shell, Ansible, git, administration système
      basique, etc
    \item Projets internes de sécurité pour référencement : automatisation du
      chiffrement de postes utilisateurs, étude de chiffrements des emails et
      authentification forte
  \end{itemize}
}

\tldatecventry{2015}{Prestataire Système/Architecture/DevOps}{OFI pour le
compte de Smile}{Paris}{5 mois}
{
  \begin{itemize}
    \item Formation git, Ansible et Linux
    \item Industrialiasation via Ansible : inventaire dynamique basé sur Active
      Directory, création de machine sur VMWare, provisionnement d'un socle de
      base, spécialisation selon le rôle de la machine, déploiement applicatif
    \item Intégration Continue et Déploiement Continu : Gitlab, Teamcity,
      Artifactory, déploiement par Ansible
    \item DevOps : rapprochement des développeurs internes et des équipes de
      production, assistance aux développeurs à l'intégration de leurs
      applications au SI
  \end{itemize}
}

\tldatecventry{2015}{Stagiaire Ingénieur Système et Infrastructure}{Smile Open
Source Solutions}{Paris}{6 mois}
{Stage de fin d'études avec pour sujet "Déploiement et gestion de postes de
travail Open Source", couvrant l'automatisation du déploiement de postes via
PXE (Cobbler) et leur gestion via SaltStack. \'{E}galement en charge du support
technique interne (OpenVZ, git, intégration, automatisation) ainsi que du
support clients (infrastructures mail, haute disponibilité, etc)}

\tlcventry{2012}{2015}{Assistant C, Unix, Ocaml, C++ et Java}{Epita}{Paris}{3
ans}
{Membre de l'équipe chargée de l'enseignement de l'informatique pratique à des
élèves de BAC+2 à BAC+3, incluant cours, travaux pratiques, ainsi que création
de sujets de projets et d'examens.}

\tlcventry{2013}{2014}{Stagiaire R\&D}{Dassault Systèmes}{Vélizy}{5 mois}
{Stage "Développement d'un connecteur ENOVIA VPM V5" visant à réécrire une
partie du serveur en C++ afin d'améliorer les performances du traitement des
requêtes ainsi que leur compatibilité avec les versions plus récentes.}

\bigskip
\subsection{Personnelle}

\tlcventry{2015}{0}{Contributions Open Source}{}{}{}
{
  \begin{itemize}
    \item Core contributeur de Py3status
    \item Code : SaltStack, Zabbix, Radicale
    \item Documentation : git-scm.com, CAS, Wallabag, KodoKojo
    \item Mainteneur de paquets sur l'AUR : py3status, py3status-git,
      python2-mygpoclient
  \end{itemize}
}

\tlcventry{2012}{0}{horgix.fr}{Administration de 3 serveurs dédiés personnels
servant de "playing ground"}{}{}
{Basé sur Docker, LXC, Gitlab CI, Mesos, Marathon, Saltstack, et hébergeant des
services personnels tels que radicale, agendav, syncthing, wallabag.}

\section{Formation}

\tlcventry{2010}{2015}{\'{E}cole d'Ingénieurs en Informatique}{EPITA}{Paris}{}{}

\tlcventry{2013}{2015}{Spécialisation SRS : Sécurité, Réseau et
Système}{EPITA}{Paris}{}{}

\tldatecventry{2012}{Semestre à l'international}{Brock University}{St
Catharines (Ontario), Canada}{5 mois}{}

\section{Compétences}

\cvitem{Développement}{\textbf{Python}, \textbf{C++}, \textbf{C}, C\#, Java,
OCaml, Shell Scripting; notions de \textbf{Haskell} et \textbf{Go}; attirance
pour le Rust}{}{}
\cvitem{Automatisation}{\textbf{Ansible},
\textbf{SaltStack}, concepts de l'Infrastructure as Code}{}{}
\cvitem{Containerisation}{\textbf{Docker}, LXC, OpenVZ}{}{}
\cvitem{Orchestration}{Mesos, Marathon}{}{}
\cvitem{Système}{\textbf{GNU/Linux} (Archlinux, CentOS, Debian, ...),
OpenStack, Ceph}{}{}
\cvitem{Infrastructure}{Conception système, systèmes distribués, haute
disponibilité (Corosync, Pacemaker, HAProxy, DRBD), infrastructure web (Apache,
Nginx, MariaDB, Varnish, \textbf{Traefik})}{}{}
\cvitem{Divers}{Versionnement, intégration continue, monitoring}{}{}

\section{Langues}

\cvlanguage{Anglais}{Courant}{Pratiqué quotidiennement, étudié 8 ans; TOEIC 920}
\cvlanguage{Allemand}{Notions}{\'{E}tudié 7 ans, peu pratiqué}

\section{Centres d'interêt}

\cvitem{Vie associative}{\textbf{GConfs} (Président), organisation de
conférences techniques}
\cvitem{}{\textbf{Prologin}, organisation d'un concours d'informatique annuel}
\cvitem{Sport}{Poweriser, squash, natation, ski}
\cvitem{Autres}{Lecture, Meetups, Rubik's cube, nouvelles technologies}

\end{document}
