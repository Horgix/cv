\documentclass[10pt,a4paper]{moderncv}

\usepackage{moderntimeline}
\usepackage[french]{babel}
\usepackage[utf8x]{inputenc}
\usepackage[T1]{fontenc}

%\renewcommand{\FrenchLabelItem}{\textcolor{blue}{$\circ$}}

\moderncvstyle{classic}
\moderncvcolor{blue}
\tlmaxdates{2010}{2016}
\tlwidth{0.8ex}
\tltext{\tiny}
% adjust the page margins
\usepackage[scale=0.90]{geometry}
\setlength{\hintscolumnwidth}{2.5cm}           % if you want to change the width of the column with the dates

% personal data
\firstname{Alexis}
\familyname{Chotard}
\title{Ingénieur Système et Infrastructure}
\address{20 rue des Jardins}{92600 Asnières-sur-Seine}
\mobile{06-08-92-41-96}
\phone{01-41-40-29-01}
\email{alexis.chotard@smile.fr}
\homepage{www.smile.fr/Offres/Infrastructure}
\extrainfo{22 ans - Permis B}
\photo[60pt][0.2pt]{photo_cv}
\quote{"People think that computer science is the art of geniuses but the
actual reality is the opposite, just many people doing things that build on
each other, like a wall of mini stones." -- Donald Knuth}


\begin{document}

\makecvtitle

\section{Formation}

\tlcventry{2013}{2015}{Spécialisation SRS : Sécurité, Réseau et
Système}{EPITA}{Paris}{}{}

\tldatecventry{2012}{Semestre à l'international}{Brock University}{St
Catharines (Ontario), Canada}{5 mois}{}

\tlcventry{2010}{2015}{\'{E}cole d'Ingénieurs en Informatique}{EPITA}{Paris}{}{}

\tldatecventry{2010}{Baccalauréat S}{Lycée Claude Gelée}{\'{E}pinal}{avec
mention}{}

\section{Expérience}

\subsection{Professionnelle}

\tldatecventry{2015}{Stagiaire ingénieur Système et Infrastructure}{Smile Open
Source Solutions}{Asnières-sur-Seine}{6 mois}
{Stage "Étude d'automatisation du déploiement et de la gestion de postes de
  travail". Technologies : Cobbler, Salt
  L'objectif de la mission est le déploiement d'un parc de postes de travail open
  sources :
  \begin{itemize}
    \item Audit de solutions actuellement en place et définition du besoin
    \item Rédaction d'un cahier des charges technique définissant les objectifs
      en terme de postes de travail open source : déploiement, bootstrap, gestion
      au quotidien, fin de vie du poste
    \item Déploiement de postes de travail et mise à disposition d'images
      \begin{itemize}
        \item Analyse et comparaison des outils du marché : FOG, FAI, OpenQRM,
          Cobbler
        \item Choix de la solution : boot PXE + Cobbler
        \item Écriture de kickstart et preseed permettant de provisionner une
          poste basique : partitionnement, chiffrement des partitions,
          installation et configuration d'un agent Salt
        \item Configuration des images à disposition
      \end{itemize}
    \item Automatisation du provisionnement de postes de travail
      \begin{itemize}
        \item Comparaison et essais de solutions de gestion de postes de travail
          : Ansible, CFEngine, Salt
        \item Choix et mise en place de la solution : Salt
      \end{itemize}
    \item Gestion de la conformité, de configurations et d'applicatifs sur un
      parc de postes de travail
      \begin{itemize}
        \item Analyse des solutions disponibles : Puppet, Ansible, Salt
        \item Choix en concordance avec les autres parties du projet : Salt
        \item Écriture de modules Salt
      \end{itemize}
    \item Rédaction de documentation à destination de Smile sur les solutions
      étudiées et sur celles finalement implémentées
  \end{itemize}
  \emph{Environnement technique : Cobbler, Salt, PXE, docker, KVM}
  \newline Missions diverses :
  \begin{itemize}
    \item Formations, notamment à destination du Ministère de l'Éducation
      Nationale
    \item Administration de serveurs pour 800+ personnes
    \item Mise en place de démonstrations à but avant-vente
    \item Organisation de présentations techniques hebdomadaires
    \item Tests de montée en charge
    \item Gestion de plateformes clientes
    \item Mise en place d'infrastructures de haute disponibilité
    \item \emph{Environnement techniaue : OpenVZ, LXC, CentOS, Debian, git,
      SVN, shell scripting, troubleshooting, Python, OpenStack, Ansible}
  \end{itemize}
}

\tlcventry{2012}{2015}{Assistant C, Unix, Ocaml, C++ et Java}{Epita}{Paris}{}
{Membre de l'équipe chargée de l'enseignement de l'informatique pratique à des
élèves de BAC+2 et BAC+3, incluant :
\begin{itemize}
\item Création de supports de cours
\item Enseignement à des amphithéâtre des 150 personnes
\item Création de supports de travaux pratiques
\item Présentation/aide aux étudiants durant les travaux pratiques
\item Création de sujets d'examens et de critères de notation
\item Maintenance évolutive de références répondant aux exigences des travaux
pratiques et examens
\item Gestion de la trésorerie
\end{itemize}
\emph{Environnement technique : C, Unix, OCaml, C++, Java}}

\tlcventry{2013}{2014}{Stagiaire R\&D}{Dassault Systèmes}{Vélizy}{5 mois}
{Stage "Développement d'un connecteur ENOVIA VPM V5" visant à réécrire une
partie du serveur afin d'améliorer les performances du traitement des requêtes
ainsi que leur compatibilité avec les versions plus récentes
\newline\emph{Environnement technique : C++ haute performance}}

\tldatecventry{2012}{Développeur-Stagiaire}{Process Engineering}{Arches}{3 mois}
{Stage "Associativité C++/C\#" ayant pour but d'étudier et de comparer les
différentes manières de faire communiquer des processus (IPC) dans ces deux
langages :
\begin{itemize}
  \item Shared memory
  \item Named pipes
  \item Sockets
\end{itemize}
\emph{Environnement technique : C++, C\#, Mercurial, WinAPI}}

\subsection{Scolaire}

\tlcventry{2012}{2014}{Tiger Compiler}{Développement d'un compilateur intégral
en C++ pour le langage Tiger}{6 mois}{4 personnes}{}

\tldatecventry{2012}{42sh}{Un interpréteur de commande (shell) POSIX-compliant
en C}{1 mois}{5 personnes}{}

\tlcventry{2010}{2011}{VorTeX}{Un jeu vidéo Portal-like en C\# utilisant
DirectX}{6 mois}{3 personnes}{}

\subsection{Personnelle}

\tlcventry{2014}{0}{Packaging Arch Linux}{Maintenance de packages sur
l'\href{https://aur.archlinux.org/packages/?O=0&SeB=m&K=horgix&outdated=&SB=n&SO=a&PP=50&do_Search=Go}{AUR}}{}{}
{
\begin{itemize}
  \item \href{https://aur.archlinux.org/packages/py3status/}{py3status}
  \item \href{https://aur.archlinux.org/packages/py3status-git/}{py3status-git}
  \item
    \href{https://aur.archlinux.org/packages/python2-mygpoclient/}{python2-mygpoclient}
\end{itemize}
}

\section{Compétences}

\cvitem{Langages}{C++, C, C\#, Python, Haskell, Java, OCaml, Shell
Scripting}{}{}
\cvitem{Divers}{Environnement Unix, \LaTeX, Versionnement (Git, Mercurial,
SVN)}{}{}

\section{Langues}
\cvlanguage{Anglais}{Courant}{Pratiqué quotidiennement, étudié 8 ans; TOEIC 920}
\cvlanguage{Allemand}{Notions}{\'{E}tudié 7 ans, peu pratiqué}

\section{Centres d'interêt}

\cvitem{Vie associative}{\textbf{GConfs} (Président), organisation de
conférences}
\cvitem{}{\textbf{Prologin}, organisation d'un concours d'informatique annuel}
\cvitem{Sport}{Poweriser, squash, natation, ski}
\cvitem{Autres}{Lecture, Rubik's cube, nouvelles technologies}

\end{document}
