\newcommand\Colorhref[3][color1]{\href{#2}{\color{#1}\underline{#3}}}
\documentclass[10pt,a4paper]{moderncv}

\usepackage{moderntimeline}
\usepackage[french]{babel}
\usepackage[utf8x]{inputenc}
\usepackage[T1]{fontenc}

%\renewcommand{\FrenchLabelItem}{\textcolor{blue}{$\circ$}}

\moderncvstyle{classic}
\moderncvcolor{blue}
\tlmaxdates{2009}{2016}
\tlwidth{0.8ex}
\tltext{\tiny}
% adjust the page margins
\usepackage[scale=0.90]{geometry}
\setlength{\hintscolumnwidth}{2.5cm}           % if you want to change the width of the column with the dates

% personal data
\firstname{Alexis}
\familyname{Chotard}
\title{Ingénieur Système \& Infrastructure\newline
DevOps à \Colorhref{http://www.smile.fr/}{Smile Open Source Solutions}}
\address{1 rue Jean Mermoz}{Appartement 157}{92320 Châtillon}
\mobile{06-08-92-41-96}
% \phone{03-29-32-71-27}
\email{alexis.horgix.chotard@gmail.com}
\homepage{https://github.com/horgix}
\extrainfo{23 ans - Permis B}
\photo[60pt][0.2pt]{photo_cv}
\quote{"People think that computer science is the art of geniuses but the
actual reality is the opposite, just many people doing things that build on
each other, like a wall of mini stones." -- Donald Knuth}

\begin{document}

\makecvtitle

\section{Expérience}

\subsection{Professionnelle}

\tlcventry{2015}{0}{Ingénieur Système et Infrastructure / DevOps}{Smile Open
Source Solutions}{Paris}{}
{Au sein de Smile en tant qu'Ingénieur Système et Infrastructure, j'ai eu
l'opportunité d'effectuer de nombreuses missions sur des technologies très
variées tout en restant dans un contexte Open Source me plaisant
particulièrement, notamment de par la possibilité de contribuer aux solutions
utilisées.\newline
Mes principales  expériences  se  regroupent  autour  de
l'industrialisation/automatisation (\textbf{Ansible}, \textbf{SaltStack}), de
la virtualisation/containerisation (\textbf{Docker}, LXC, Proxmox, OpenVZ),
ainsi que de l'accompagnement à la mise en place d'architectures web.
  \begin{itemize}
    \item Nombreuses missions courtes de quelques jours, par exemple :
      \begin{itemize}
        \item Automatisation via Ansible du déploiement de serveurs Magento
          (frontaux + bases de données) avec environnements de développement,
          d'intégration et de production
        \item Tests de montée en charge (JMeter)
        \item Formations : scripting shell, Ansible, git, administration
          système basique, etc
        \item Mise en place d'un Owncloud en haute disponibilité : Corosync,
          Pacemaker, HAProxy, csync2, DRBD
        \item Gestion d'un projet de générateur de plateformes de démonstration
          : chiffrage, plan de déroulement, etc. Ansible, Tower, AWS
        \item Automatisation de déploiement JBoss et Oracle
      \end{itemize}
    \item Entretiens techniques de candidats, intégration et formation des
      nouveaux arrivants et coaching de stagiaires
    \item Organisation de lightning talks hebdomadaires : UEFI, grsecurity,
      FreeIPA, Syncthing, Ansible, etc
    \item Projets internes de sécurité pour référencement : automatisation du
      chiffrement de postes utilisateurs, étude de chiffrements des emails, et
      authentification forte
    \item Vulgarisation technique : présentation "devops for sales(wo)men",
      enjeux et bénéfices de l'industrialiation et l'infrastructure as code,
      etc
    \item Rédaction de laïus commerciaux, par exemple sur
      \Colorhref{http://infrastructure.smile.eu/Notre-offre/Nos-expertises/Devops-industrialisation}{l'industrialisation
      logicielle}
    \item Avant vente, qualification, études, mise en place de plateformes
      de démonstration
  \end{itemize}
}

\tldatecventry{2016}{Prestataire Système/Architecture/DevOps}{OFI pour le
compte de Smile}{Paris}{5 mois}
{
  \begin{itemize}
    \item Formation git, Ansible et Linux
    \item Industrialiasation via Ansible
      \begin{itemize}
        \item Inventaire dynamique avec comme source Active Directory
        \item Automatisation du déploiement de machines sur VMWare
        \item Provisionnement de base des machines : monitoring (Zabbix),
          logging centralisé (Splunk), authentification (SSSD), configuration
          système, etc.
        \item Spécialisation : front, bases de données, etc
        \item Déploiement applicatif par environnement (développement,
          intégration, production)
        \item Rôles génériques
      \end{itemize}
    \item Intégration Continue / Déploiement Continu
      \begin{itemize}
        \item Migration de gitolite à Gitlab
        \item Intégration avec Jira
        \item Débuts d'intégration continue via Teamcity et Artifactory
        \item Déploiement continu via Ansible
      \end{itemize}
    \item DevOps
      \begin{itemize}
        \item Rapprochement des développeurs internes et des équipes de
          production
        \item Assistance aux développeurs afin de rendre les applications
          déployables automatiquement, proprement, et de manière intégrée au SI
        \item "\'{E}vangelisation", bonnes pratiques, documentation
          (Confluence), etc
      \end{itemize}
  \end{itemize}
}

\tldatecventry{2016}{Prestataire Système/DevOps}{AFNOR pour le compte de
Smile}{Paris}{1 semaine}
{
  \begin{itemize}
    \item Formation git et Ansible
    \item Industrialisation via Ansible :
      \begin{itemize}
        \item Développement d'un inventaire dynamique via le DNS : deléguation
          de zone + request AXFR
        \item Automatisation de la création de machines sur VMWare
        \item Création de rôles de services selon des prérequis donnés
        \item Automatisation du déploiement de Wordpress
      \end{itemize}
  \end{itemize}
}

\tldatecventry{2016}{Prestataire Système}{CCFD pour le compte de
Smile}{Paris}{2 semaine}
{
  \begin{itemize}
    \item Ansible : création d'un inventaire dynamique s'appuyant sur Proxmox
      (catégorisation par pool, état, remontée d'infos, etc.) et de rôles
      basiques
    \item Remise à plat de l'infrastructure sous-jacente : ntp, dns, pfsense,
      etc
    \item Ceph : remise en état de marche d'un cluster existant (monitors
      desynchronisés, OSDs downs, pools inutilisés, placement des PGs non
      optimal, ...) + benchmark
    \item Reconfiguration de hosts Proxmox : corosync, cluster pve
    \item Visioconférence via JitsiMeet, génération et signature d'addons
      Firefox/Chrome
  \end{itemize}
}

\tldatecventry{2016}{Prestataire Système}{Expertise France pour le compte de
Smile}{Paris}{3 mois}
{
  \begin{itemize}
    \item Automatisation de la configuration des serveurs via Ansible
    \item Formation git et Ansible
    \item Migration de plateforme mail (Sogo vers Bluemind)
    \item Mise en place de visioconférence via JitsiMeet (Prosody + Jitsi
      videobridge + jicofo)
    \item Développement d'un webservice de génération d'invitations et d'accès
      pour la visioconférence : backend en python, frontend responsive,
      génération d'emails à partir de templates Jinja2
    \item Mise en place d'un cluster Ceph (Infernalis) : installation des
      serveurs, configuration, tuning de la crushmap, des règles de placement,
      des pools, et benchmark
    \item Mise en place de SSO via CAS (Kerberos + fallback LDAP)
  \end{itemize}
}

\tldatecventry{2016}{Prestataire Système}{Groupe Alpha pour le compte de
Smile}{Paris}{2 semaines}
{
  \begin{itemize}
    \item Homogénéisation et centralisation des logs : ElasticSearch / Logstash
      / Kibana
    \item Authentification Kerberos via SSSD
    \item Automatisation du déploiement des configurations via Ansible
  \end{itemize}
}

\tldatecventry{2015}{Stagiaire Ingénieur Système et Infrastructure}{Smile Open
Source Solutions}{Paris}{6 mois}
{Stage de fin d'études avec pour sujet "Déploiement et gestion de postes de
travail Open Source", couvrant l'automatisation du déploiement de postes via
PXE et leur gestion (inventaire, contrôle à distance, gestion de configuration,
etc).
\begin{itemize}
  \item \'{E}tude initiale sur Pulse2 (développé par Mandriva) s'étant soldée
    de manière non convaincante et par 25 pages de bug report 3 mois avant le
    dépôt de bilan de Mandriva
  \item \'{E}tudes succinte et comparaison de Katello, Foreman, Cobbler, FOG,
    OpenQRM, FAI, CFEngine, Ansible, Rudder, Puppet, Chef, SaltStack, etc.
  \item Solution retenue, maquettée et validée : \textbf{Cobbler + SaltStack}
\end{itemize}
\'{E}galement en charge de :
\begin{itemize}
  \item Support interne : OpenVZ, git, SVN, intégration, automatisation, debug
    système, etc
  \item Support clients : infrastructures mail, haute disponibilité, etc
\end{itemize}
}

\tlcventry{2012}{2015}{Assistant C, Unix, Ocaml, C++ et Java}{Epita}{Paris}{3
ans}
{Membre de l'équipe chargée de l'enseignement de l'informatique pratique à des
élèves de BAC+2 à BAC+3, incluant cours, travaux pratiques, ainsi que création
de sujets de projets et d'examens.}

\tlcventry{2013}{2014}{Stagiaire R\&D}{Dassault Systèmes}{Vélizy}{5 mois}
{Stage "Développement d'un connecteur ENOVIA VPM V5" visant à réécrire une
partie du serveur en C++ afin d'améliorer les performances du traitement des
requêtes ainsi que leur compatibilité avec les versions plus récentes.}

\tldatecventry{2012}{Stagiaire développeur}{Process Engineering}{Arches}{3 mois}
{Stage "Associativité C++/C\#" ayant pour but d'étudier et de comparer les
différentes manières de faire communiquer des processus (IPC) dans ces deux
langages.}

\bigskip
\subsection{Personnelle}

\tlcventry{2015}{0}{Contributions Open Source}{}{}{}
{
  \begin{itemize}
    %\item \Colorhref{https://github.com/Horgix?tab=repositories}{Github}
    %\item \Colorhref{https://www.openhub.net/accounts/Horgix/positions}{OpenHub}
    \item Core contributeur de Py3status
    \item Code : saltstack, zabbix, radicale
    \item Documentation : git-scm.com, CAS, wallabag, kodokojo
    \item Mainteneur de paquets sur l'\Colorhref{https://aur.archlinux.org}{AUR} :
      \Colorhref{https://aur.archlinux.org/packages/py3status/}{py3status},
      \Colorhref{https://aur.archlinux.org/packages/py3status-git/}{py3status-git},
      \Colorhref{https://aur.archlinux.org/packages/python2-mygpoclient/}{python2-mygpoclient},
      \Colorhref{https://aur.archlinux.org/packages/terminal-parrot/}{terminal-parrot}
  \end{itemize}
}

\tlcventry{2012}{0}{horgix.fr}{Administration de 3 serveurs dédiés personnels
servant de "playing ground"}{}{}
{\begin{itemize}
  \item Containerisation : Docker, LXC
  \item Intégration Continue : Gitlab, Gitlab CI
  \item Orchestration : Mesos, Marathon
  \item Industrialisation : SaltStack
  \item Services personnels : Radicale, Agendav, Syncthing, Wallabag, horgix.fr
\end{itemize}
}

\subsection{Scolaire}

\tlcventry{2012}{2014}{Tiger Compiler}{Développement d'un compilateur intégral
en C++ pour le langage Tiger}{6 mois}{4 personnes}{}

\tldatecventry{2012}{42sh}{Un interpréteur de commandes (shell) POSIX-compliant
en C}{1 mois}{5 personnes}{}

\tlcventry{2010}{2011}{VorTeX}{Un jeu vidéo Portal-like en C\# utilisant
DirectX}{6 mois}{3 personnes}{}

\section{Formation}

\tlcventry{2010}{2015}{\'{E}cole d'Ingénieurs en Informatique}{EPITA}{Paris}{}{}

\tlcventry{2013}{2015}{Spécialisation SRS : Sécurité, Réseau et
Système}{EPITA}{Paris}{}{}

\tldatecventry{2012}{Semestre à l'international}{Brock University}{St
Catharines (Ontario), Canada}{5 mois}{}

\tldatecventry{2010}{Baccalauréat S}{Lycée Claude Gelée}{\'{E}pinal}{avec
mention} {}

\section{Compétences}

\cvitem{Développement}{\textbf{Python}, \textbf{C++}, \textbf{C}, C\#, Java,
OCaml, Shell Scripting; notions de \textbf{Haskell} et \textbf{Go}; attirance
pour le Rust}{}{}
\cvitem{Industrialisation, Intégration}{\textbf{Ansible},
\textbf{SaltStack}}{}{}
\cvitem{Containerisation, orchestration}{Docker, LXC, Mesos, Marathon}{}{}
\cvitem{Système}{GNU/Linux (Archlinux, CentOS, Debian, ...), OpenStack,
Ceph}{}{}
\cvitem{Infrastructure}{Conception système, haute disponibilité, systèmes
distribués}{}{}
\cvitem{Divers}{Versionnement, intégration continue, monitoring, etc}{}{}

\section{Langues}

\cvlanguage{Anglais}{Courant}{Pratiqué quotidiennement, étudié 8 ans; TOEIC 920}
\cvlanguage{Allemand}{Notions}{\'{E}tudié 7 ans, peu pratiqué}

\section{Centres d'interêt}

\cvitem{Vie associative}{\textbf{\Colorhref{https://gconfs.fr/}{GConfs}}
(Président), organisation de conférences techniques}
\cvitem{}{\textbf{\Colorhref{https://prologin.org/}{Prologin}}, organisation d'un
concours d'informatique annuel}
\cvitem{Sport}{Poweriser, squash, natation, ski}
\cvitem{Autres}{Lecture, Meetups, Rubik's cube, nouvelles technologies}

\end{document}
