\documentclass[11pt,a4paper]{moderncv}

\usepackage{moderntimeline}
\usepackage[french]{babel}
\usepackage[utf8x]{inputenc}
\usepackage[T1]{fontenc}

\renewcommand{\FrenchLabelItem}{\textcolor{blue}{$\circ$}}

\moderncvstyle{classic}
\moderncvcolor{blue}
\tlmaxdates{2009}{2013}
\tlwidth{0.8ex}
\tltext{\tiny}
% adjust the page margins
\usepackage[scale=0.90]{geometry}
\setlength{\hintscolumnwidth}{2.5cm}           % if you want to change the width of the column with the dates

% personal data
\firstname{Alexis}
\familyname{Chotard}
\title{\'{E}tudiant à Epita}
\address{32 Route de Remiremont}{88380 ARCHES}
\mobile{(+33)6-08-92-41-96}
\phone{(+33)3-29-32-71-27}
\email{alexis.horgix.chotard@gmail.com}
\homepage{www.horgix.fr}
\quote{"People think that computer science is the art of geniuses but the actual reality is the opposite, just many people doing things that build on each other, like a wall of mini stones." (Donald Knuth)}


\begin{document}

\makecvtitle

\section{\'{E}ducation}

\tlcventry{2010}{0}{Ingénierie informatique}{EPITA}{Paris}{}
{Programmation (C, OCaml, C++, C\#, ...), Algorithmique, Physique, Droit,
    Architecture des Systèmes, \'{E}lectronique,
    Mathématiques, Systèmes d'exploitation, Marketing, ... }

\tldatecventry{2012}{Semestre à l'international}{Brock University}{St
Catharines, Ontario, Canada}{5 mois}
{\'{E}conomie, \'{E}lectronique, Mathématiques, Programmation (C), ...}

\tldatecventry{2010}{Baccalauréat S}{Lycée Claude Gelée}{Epinal}{Mention assez bien}
{}

\section{\'{E}xpérience}

\subsection{Professionnelle}

\tlcventry{2012}{0}{Assistant OCaml/C}{Epita}{Paris}{}
{Professeur lors de Travaux Pratiques pour des élèves à BAC+2 sur les sujets
    suivants : Langage C et OCaml, Environnement Unix, Organisation de projet.}

\tldatecventry{2012}{Développeur-Stagiaire}{Process Engineering}{Arches}{3 mois}
{Stage "Association C++/C\#" visant à étudier et comparer les différentes
    manières de faire communiquer des processus (IPC) dans ces deux langages
    (Shared Memory, Sockets, Named Pipes, DLL)}

\subsection{Scolaire}

\tldatecventry{2012}{42sh}{Un intérpreteur de commande (shell) en C imitant
'bash --posix'}{1 mois, dans un groupe de 5 personnes}{}{}

\tlcventry{2010}{2011}{VorTeX}{Un jeu vidéo Portal-like en C\# utilisant
DirectX}{}{}{}

\subsection{Personnelle}

\tlcventry{2012}{0}{My MIPS}{Un émulateur MIPS en C++}{}{}{}

\tlcventry{2012}{0}{Tiger Compiler}{Un compilateur Tiger entier, écrit en C et
utilisant Flex et Bison pour le lexer/parser}{}{}{}

\tlcventry{2012}{0}{My server}{Administration d'un serveur dédié personnel,
ayant pour but d'acquérir la certification IPv6 d'Hurricane Electric}{}{}{}

\section{Compétences}

\cvcomputer{Langages}{C, C++, C\#, OCaml, Python, Shell/Bash, awk, sed, GNU Make}
{Systèmes d'exploitqtion}{GNU/Linux (Archlinux, Fedora, Opensuse), Windows (XP,
Seven)}

\cvcomputer{Web}{HTML, CSS}{Méthodes}{Développement Orienté Objet, UML,
MindMaps}
\cvcomputer{Gestion de sources}{SVN, Git, Mercurial}
{Outils}{Doxygen, Dial}

\section{Langues}
\cvlanguage{Français}{Natif}{Langue natale}
\cvlanguage{Anglais}{Courant}{Pratiqué quotidiennement, étudié 8 ans}
\cvlanguage{Allemand}{Notions}{\'{E}tudié 7 ans, peu pratiqué}

\section{Centres d'interêt}

\cvitem{Vie associative}{\textbf{GConfs} (Trésorier), organisation de conférences}
\cvitem{}{\textbf{Prologin}, organisation d'un concours d'informatique annuel}
\cvitem{}{\textbf{AEDD}, association pour le développement durable}
\cvitem{Sport}{Poweriser, squash, natation, ski}
\cvitem{Autres}{Lecture, Rubik's cube, nouvelles technologies}

\end{document}
