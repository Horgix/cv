\documentclass[11pt,a4paper]{moderncv}

\usepackage{moderntimeline}
\usepackage[french]{babel}
\usepackage[utf8x]{inputenc}
\usepackage[T1]{fontenc}

%\renewcommand{\FrenchLabelItem}{\textcolor{blue}{$\circ$}}

\moderncvstyle{classic}
\moderncvcolor{blue}
\tlmaxdates{2009}{2014}
\tlwidth{0.8ex}
\tltext{\tiny}
% adjust the page margins
\usepackage[scale=0.90]{geometry}
\setlength{\hintscolumnwidth}{2.5cm}           % if you want to change the width of the column with the dates

% personal data
\firstname{Alexis}
\familyname{Chotard}
\title{\'{E}tudiant à Epita, \newline \textit{\Large 2\up{ème} année de cycle
Ingénieur}} %\textit {\Large Recherche un stage de 4
%mois de\newline 2\up{ème} année de cycle Ingénieur à\newline  partir de début Septembre 2013}}
\address{32 Route de Remiremont}{88380 ARCHES}
\mobile{06-08-92-41-96}
\phone{03-29-32-71-27}
\email{alexis.horgix.chotard@gmail.com}
\homepage{www.horgix.fr}
\extrainfo{20 ans - Permis B}
\photo[60pt][0.2pt]{photo_cv}
\quote{"People think that computer science is the art of geniuses but the
actual reality is the opposite, just many people doing things that build on
each other, like a wall of mini stones." -- Donald Knuth}


\begin{document}

\makecvtitle

\section{Formation}

\tlcventry{2010}{0}{\'{E}cole d'Ingénieurs en Informatique}{EPITA}{Paris}{}{}

\tldatecventry{2012}{Semestre à l'international}{Brock University}{St
Catharines (Ontario), Canada}{5 mois}{}

\tldatecventry{2010}{Baccalauréat S}{Lycée Claude Gelée}{\'{E}pinal}{avec mention}
{}

\section{Expérience}

\subsection{Professionnelle}

\tlcventry{2013}{0}{Assistant C++/Java}{Epita}{Paris}{}
{Membre de l'équipe chargée de l'enseignement de la programmation en C++ et
Java à des élèves à BAC+3; incluant cours, Travaux Pratiques, ainsi que
création de sujets de projets et d'examens.}
\tlcventry{2013}{2014}{Stagiaire R\&D}{Dassault Systèmes}{Vélizy}{5 mois}
{Stage "Développement d'un connecteur ENOVIA VPM V5" visant à réécrire une
partie du serveur afin d'améliorer les performances du traitement des requêtes
ainsi que leur compatibilité avec les versions plus récentes.}
\tlcventry{2012}{2013}{Assistant OCaml/C}{Epita}{Paris}{1 an}
{Assistant lors de Travaux Pratiques sur les langages C et OCaml ainsi que sur
l'environnement Unix.}

\tldatecventry{2012}{Développeur-Stagiaire}{Process Engineering}{Arches}{3 mois}
{Stage "Associativité C++/C\#" ayant pour but d'étudier et de comparer les
différentes manières de faire communiquer des processus (IPC) dans ces deux
langages.}

\subsection{Scolaire}

\tlcventry{2012}{0}{Tiger Compiler}{Développement d'un compilateur intégral en
C++ pour le langage Tiger}{}{}{}

\tldatecventry{2012}{42sh}{Un interpréteur de commande (shell) POSIX-compliant
en C}{1 mois}{5 personnes}{}

\tlcventry{2010}{2011}{VorTeX}{Un jeu vidéo Portal-like en C\# utilisant
DirectX}{6 mois}{3 personnes}{}

\section{Compétences}

\cvitem{Langages}{C++, C, C\#, Python, Java, OCaml, Shell Scripting}{}{}
\cvitem{Divers}{Environnement Unix, \LaTeX, Versionnement (Git, Mercurial, SVN)}{}{}

\section{Langues}
\cvlanguage{Anglais}{Courant}{Pratiqué quotidiennement, étudié 8 ans; TOEIC 920}
\cvlanguage{Allemand}{Notions}{\'{E}tudié 7 ans, peu pratiqué}

\section{Centres d'interêt}

\cvitem{Vie associative}{\textbf{GConfs} (Vice-Président), organisation de conférences}
\cvitem{}{\textbf{Prologin}, organisation d'un concours d'informatique annuel}
\cvitem{Sport}{Poweriser, squash, natation, ski}
\cvitem{Autres}{Lecture, Rubik's cube, nouvelles technologies}

\end{document}
