\documentclass[11pt,a4paper]{moderncv}

\usepackage{moderntimeline}
\usepackage[french]{babel}
\usepackage[utf8x]{inputenc}
\usepackage[T1]{fontenc}

\renewcommand{\FrenchLabelItem}{\textcolor{blue}{$\circ$}}

\moderncvstyle{classic}
\moderncvcolor{blue}
\tlmaxdates{2009}{2013}
\tlwidth{0.8ex}
\tltext{\tiny}
% adjust the page margins
\usepackage[scale=0.85]{geometry}
\setlength{\hintscolumnwidth}{2.5cm}           % if you want to change the width of the column with the dates

% personal data
\firstname{Alexis}
\familyname{Chotard}
\title{\'{E}tudiant à Epita}
\address{32 Route de Remiremont}{88380 ARCHES}
\mobile{(+33)6-08-92-41-96}
\phone{(+33)3-29-32-71-27}
\email{alexis.horgix.chotard@gmail.com}
\homepage{www.horgix.fr}
\quote{"People think that computer science is the art of geniuses but the actual reality is the opposite, just many people doing things that build on each other, like a wall of mini stones." (Donald Knuth)}


\begin{document}

\makecvtitle

\section{\'{E}ducation}

\tlcventry{2010}{0}{Ingénierie informatique}{EPITA}{Paris}{}
{Programmation (C, OCaml, C++, ...), Algorithmique, Physique, Droit,
    Architecture des Systèmes, \'{E}lectronique,
    Mathématiques,Théorie des Languages, Systèmes d'exploitation,
    Marketing, ... }

\tldatecventry{2012}{Semestre à l'international}{Brock University}{St
Catharines, Ontario, Canada}{5 mois}
{\'{E}conomie, \'{E}lectronique, Mathématiques, Programmation (C), ...}

\tldatecventry{2010}{Baccalauréat S}{Lycée Claude Gelée}{Epina}{Mention assez bien}
{}

\section{\'{E}xpérience professionnelle}

\tlcventry{2012}{0}{Assistant OCaml/C}{Epita}{Paris}{}
{Professeur lors de Travaux Pratiques pour des élèves à BAC+2 sur les sujets suivants :
\begin{itemize}
    \item{Langage C}
    \item{Langage Ocaml}
    \item{Environnement Unix}
    \item{Organisation de projet}
\end{itemize}
}

\tldatecventry{2012}{Développeur-Stagiaire}{Process Engineering}{Arches}{3 mois}
{Stage "Association C++/C\#" visant à étudier les différentes manières de faire
    communiquer des processus (IPC) dans ces deux langages au moyen des
    diverses solutions :
\begin{itemize}
    \item{Shared Memory}
    \item{Sockets}
    \item{Named Pipes}
    \item{DLLs}
\end{itemize}
}

\section{Compétences}

\section{Langues}
\cvlanguage{Français}{Natif}{Langue natale}
\cvlanguage{Anglais}{Courant}{Pratiqué quotidiennement, étudié 8 ans}
\cvlanguage{Allemand}{Notions}{\'{E}tudié 7 ans, peu pratiqué}

\section{Centres d'interêt}

\cvitem{Vie associative}{\textbf{GConfs} (Trésorier), organisation de conférences}
\cvitem{}{\textbf{Prologin}, organisation d'un concours d'informatique annuel}
\cvitem{}{\textbf{AEDD}, association pour le développement durable}
\cvitem{Sport}{Poweriser, squash, natation, ski}
\cvitem{Autres}{Lecture, Rubik's cube, nouvelles technologies}

\end{document}
